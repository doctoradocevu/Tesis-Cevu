avalanchas 

Por otro lado, los experimentos a escalas más pequeñas (microscópicas) se han concentrado en la descripción de la actividad cerebral como eventos de avalancha discretos, que se propagan a través de la corteza de forma sin escala [12]. Estas avalanchas sin escala se han expuesto mediante técnicas electrofisiológicas en diferentes entornos [7, 13, 14]. Se sabe que los sistemas críticos autoorganizados disipan energía en forma de avalanchas distribuidas según una ley de potencias [15, 2], por lo que esta es una evidencia directa que favorece la hipótesis de que el cerebro alcanza propiedades críticas a través de la autorregulación y no requiere un ajuste fino de los parámetros. Volvemos a enfatizar que los experimentos a esta escala, ya sea grabaciones de unidades individuales o de potenciales de campo local (LFP), demuestran intermitencia y estallidos sin escala, con la intensidad del estallido discreto obedeciendo una distribución de ley de potencias.


