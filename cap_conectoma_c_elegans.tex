En las últimas dos décadas, la ciencia de redes ha florecido e influido en varios campos, como la física estadística, la informática, la biología y la sociología, desde la perspectiva de los patrones de interacción heterogéneos de los componentes que componen los sistemas complejos.


A  diferencia de muchos otros campos de investigación modernos, el problema de las redes suele ser fácil de definir abstrayéndose de la vida cotidiana. Por ejemplo, cuántas personas puede infectar una epidemia en una red de contactos sociales, si una red de comunicación puede mantener su función tras un ataque intencionado, qué nodo tiene el mayor impacto en una red social, etc. 
