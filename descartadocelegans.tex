

El nematodo Caenorhabditis elegans, de un milímetro de longitud, transparente y con un ciclo de vida de 3,5 días a 20 °C, es un modelo favorito para los estudios de comportamiento debido a su sistema nervioso compacto y su tratabilidad experimental. Estas ventajas permiten diseccionar el comportamiento a nivel de genes, neuronas individuales y circuitos neuronales. El entorno natural de C. elegans es el suelo, donde se alimenta de bacterias y otros microbios. Se mueve propagando curvas a lo largo de su cuerpo, de forma muy similar a una serpiente, y se orienta en su entorno utilizando pistas mecánicas, térmicas y, sobre todo, químicas. El repertorio de comportamientos de este organismo sencillo es sorprendentemente rico, y como muchos de sus comportamientos pueden cuantificarse fácilmente, se pueden identificar y clasificar rápidamente mutantes genéticos informativos \cite{bono_neuronal_2005}. 









C. elegans se presenta en dos sexos muy dimórficos: machos y hermafroditas autofecundantes. La mayoría de los estudios de comportamiento se han centrado en los hermafroditas, con la excepción de los estudios de apareamiento. Los adultos de ambos sexos están compuestos por un número preciso de células: los hermafroditas tienen 959 núcleos somáticos y los machos tienen 1031. Estas células forman la hipodermis, el músculo, el tracto digestivo, la gónada y el sistema nervioso. En los hermafroditas, el sistema nervioso está formado por 302 neuronas y 56 células gliales y de soporte, mientras que los machos tienen 381 neuronas y 92 células gliales y de soporte. Aproximadamente la mitad de los cuerpos celulares neuronales se encuentran en la cabeza, rodeando un neuropilo central llamado anillo nervioso. El resto se encuentra a lo largo del cordón ventral y en los ganglios de la cola. Las neuronas específicas de los machos se encuentran principalmente en la cola copulatoria. En ambos sexos, cada neurona es reconocible de forma única en diferentes individuos por su posición y morfología características.