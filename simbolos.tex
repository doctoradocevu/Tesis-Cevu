
\glsxtrnewsymbol
[description={Conjunción lógica and (la proposición $A \land B$ es verdadera si $A$ y $B$ son ambas verdaderas; de otra manera es falsa)}]% options
{and}% label
{\ensuremath{\land}}% symbol



\glsxtrnewsymbol
[description={Disyunción exclusiva XOR (la proposición $A \veebar B$  es verdadera si y solo si una es verdadera y la otra es falsa; si las dos frases son ambas verdaderas o ambas falsas, la proposición es falsa)}]% options
{xor}% label
{\ensuremath{\veebar}}% symbol

\glsxtrnewsymbol
[description={Cuantificador universal  ($\forall x : P(x)$ significa: $P(x)$ es verdadera \textbf{para todo} $x$)}]% options
{forall}% label
{\ensuremath{\forall}}% symbol

\glsxtrnewsymbol
[description={Cuantificador existencial  ($\exists x : P(x)$ significa: \textbf{existe por lo menos un} $x$ tal que $P(x)$ es verdadera)}]% options
{existe}% label
{\ensuremath{\exists}}% symbol

\glsxtrnewsymbol
[description={Definición (definir el lado izquierdo por la expresión del lado derecho)}]% options
{definicionizquierdo}% label
{\ensuremath{:=}}% symbol

\glsxtrnewsymbol
[description={Definición a la inversa  (definir el lado derecho por la expresión del lado izquierdo)}]% options
{definicionderecho}% label
{\ensuremath{=:}}% symbol


\glsxtrnewsymbol
[description={Diferencia de conjuntos ($A \backslash B$ significa: el conjunto que contiene todos aquellos elementos de $A$ que no se encuentran en $B$)}]% options
{Diferenciaconjunts}% label
{\ensuremath{\backslash}}% symbol

\glsxtrnewsymbol
[description={Reluz  ($\exists x : P(x)$ significa: existe por lo menos un $x$ \textbf{tal que} $P(x)$ es verdadera)}]% options
{reluz}% label
{\ensuremath{:}}% symbol

\glsxtrnewsymbol
[description={Equivalencia ($ x\sim y$ significa: $x$ e $y$ son objetos, iguales o diferentes, miembros de un conjunto de objetos con la característica común de los miembros del conjunto)}]% options
{equivalencia}% label
{\ensuremath{\sim}}% symbol

\glsxtrnewsymbol
[description={Delimitadores de conjunto ($\left\{a,b,c\right\}$ significa: el conjunto que contiene a, b, y c)}]% options
{delimitadores}% label
{\ensuremath{\left\{,\right\}}}% symbol

\glsxtrnewsymbol
[description={Notación constructora de conjuntos ($\left\{x : P(x)\right\}$ significa: \textbf{el conjunto de todos} los x \textbf{tales que} $P(x)$ es verdadera. $\left\{x \mid P(x)\right\}$ es lo mismo que $\left\{x : P(x)\right\}$)}]% options
{constructora}% label
{\ensuremath{\left\{:\right\}}}% symbol


\glsxtrnewsymbol
[description={Conjunto vacío ($\emptyset$ significa: el conjunto que no tiene elementos; $\left\{\right\}$ es la misma cosa)}]% options
{Conjuntovacio}% label
{\ensuremath{\emptyset}}% symbol
	
\glsxtrnewsymbol
[description={Pertenencia de conjuntos ($a \in S$ significa: $a$ \textbf{es elemento} del conjunto $S$)	}]% options
{pertenencia}% label
{\ensuremath{\in}}% symbol

\glsxtrnewsymbol
[description={Pertenencia de conjuntos ($a \notin S$ significa: $a$ \textbf{no es elemento} del conjunto $S$)	}]% options
{nopertenencia}% label
{\ensuremath{\notin}}% symbol

\glsxtrnewsymbol
[description={subconjunto ($A \subset B$ significa: $A \subseteq B$ pero $A \neq B$)	}]% options
{subconjunto}% label
{\ensuremath{\subset}}% symbol

\glsxtrnewsymbol
[description={subconjunto ($A \subseteq B$ significa: cada elemento de $A$ es también elemento de $B$)	}]% options
{subconjuntoigual}% label
{\ensuremath{\subseteq}}% symbol

\glsxtrnewsymbol
[description={Correspondencia funcional	 ( $f: X \rightarrow Y$ significa: la función $f$ con correspondencia de $X$ en $Y$ (que va del conjunto $X$ al conjunto $Y$))	}]% options
{correspondencia}% label
{\ensuremath{f:X\rightarrow Y}}% symbol

\glsxtrnewsymbol
[description={Números naturales}]% options
{naturales}% label
{\ensuremath{\mathbb{N}}}% symbol


\glsxtrnewsymbol
[description={Números reales}]% options
{reales}% label
{\ensuremath{\mathbb{R}}}% symbol


\glsxtrnewsymbol
[description={Norma ($\parallel x \parallel$ es la norma del elemento $x$ de un espacio vectorial normado) }]% options
{norma}% label
{\ensuremath{\parallel \parallel}}% symbol

\glsxtrnewsymbol
[description={Sumatoria ($\sum_{k=1}^{n}a_k$ significa: $a_1 + a_2 + \cdots + a_n$) }]% options
{sumatoria}% label
{\ensuremath{\sum}}% symbol

\glsxtrnewsymbol
[description={Operador laplaciano (Se hace uso de los simbolos $\nabla\cdot\nabla$, $\nabla^2$, $\Delta$) }]% options
{laplaciano}% label
{\ensuremath{\nabla\cdot\nabla}}% symbol


\glsxtrnewsymbol
[description={Productoria ($\prod_{k=1}^{n}a_k$ significa: $a_1a_2\cdots a_n$) }]% options
{productoria}% label
{\ensuremath{\prod}}% symbol

\glsxtrnewsymbol
[description={Producto directo}]% options
{directo}% label
{\ensuremath{\times}}% symbol

\glsxtrnewsymbol
[description={Suma directa}]% options
{sumadirecta}% label
{\ensuremath{\oplus}}% symbol


\glsxtrnewsymbol
[description={Producto tensorial}]% options
{productotensiorial}% label
{\ensuremath{\otimes}}% symbol


\glsxtrnewsymbol
[description={Disyunción lógica inclusiva OR (la proposición $A \lor B$ es verdadera si $A$ o $B$ (o ambas) son verdaderas; si ambas son falsas, la proposición es falsa)}]% options
{or}% label
{\ensuremath{\lor}}% symbol