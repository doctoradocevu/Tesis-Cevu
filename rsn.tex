https://www.nature.com/articles/s41598-022-07860-7#ref-CR1

La simulación de la dinámica del cerebro en estado de reposo a través de modelos matemáticos de todo el cerebro requiere una selección óptima de parámetros, que determinan la capacidad del modelo para replicar datos empíricos. 

Se han desarrollado una multitud de enfoques de estudio y aplicaciones de la dinámica del estado de reposo. Su objetivo era comprender la arquitectura y el funcionamiento del cerebro. Los modelos dinámicos basados ​​en datos permiten la incorporación de información anatómica sobre el cerebro humano en la simulación de sus propiedades dinámicas. En otras palabras, permiten a los investigadores investigar la relación entre la estructura y la función del cerebro con un enfoque particular en la cuestión de si y cómo la última emerge de la primera y cómo se correlacionan.  Otra ventaja del enfoque de modelado discutido es que permite una descripción significativa de fenómenos naturales complejos mediante un conjunto de parámetros interpretables. Los modelos de cerebro completo están diseñados para imitar la dinámica del cerebro humano de una manera que reduce bastante su complejidad y lo hace accesible para análisis profundos y sistemáticos. Este procedimiento, a menudo denominado búsqueda en cuadrícula , se ha aplicado en una variedad de estudios basados ​​en modelos

El modelado dinámico de la dinámica del cerebro en estado de reposo se basa esencialmente en los datos empíricos de neuroimagen utilizados para la derivación y validación del modelo. Hace unos 15 años, se introdujo el conectoma del cerebro humano para comprender los estados cerebrales funcionales que surgen de la arquitectura estructural ( Sporns et al., 2005 ). Durante más de una década, los investigadores han estado investigando el conectoma humano para dilucidar la relación entre estructura y función. 

https://direct.mit.edu/netn/article/5/3/798/102384/Parcellation-induced-variation-of-empirical-and

Los desarrollos recientes de modelos de cerebro completo han demostrado su potencial al investigar la actividad cerebral en estado de reposo. 

La relación estructura-función en el cerebro humano ha sido un tema de interés en muchos estudios de neuroimagen ( Suárez, Markello, Betzel, & Misic, 2020 ). Aquí, la conectividad estructural (SC) y la conectividad funcional (FC), que reflejan las conexiones físicas y los patrones de coactivación sincronizada en todo el cerebro, respectivamente, no exhiben una asociación perfecta ( Honey et al., 2009 ). Un esfuerzo para cerrar esta brecha en la relación estructura-función implica el empleo de modelos dinámicos de cerebro completo.que utilizan SC como conocimiento previo para simular la actividad cerebral en estado de reposo ( Honey et al., 2009 ).
También demuestran que el cerebro en reposo opera en un estado de máxima metaestabilidad ( Deco, Kringelbach, Jirsa y Ritter, 2017 ). 

https://www.sciencedirect.com/science/article/pii/S0924977X10000684?via%3Dihub


Nuestro cerebro es una red. Consiste en regiones distribuidas espacialmente, pero vinculadas funcionalmente, que continuamente comparten información entre sí. Curiosamente, los avances recientes en la adquisición y análisis de datos de neuroimagen funcional han catalizado la exploración de la conectividad funcional en el cerebro humano. La conectividad funcional se define como la dependencia temporal de los patrones de activación neuronal de las regiones cerebrales anatómicamente separadas y, en los últimos años, un cuerpo cada vez mayor de estudios de neuroimagen ha comenzado a explorar la conectividad funcional midiendo el nivel de coactivación de las series temporales de IRMf en estado de reposo.

Nuestro cerebro es una red. Una red muy eficiente para ser precisos. Es una red de un gran número de regiones cerebrales diferentes, cada una de las cuales tiene su propia tarea y función, pero que comparten información continuamente entre sí. Como tales, forman una red integradora compleja en la que la información se procesa y transporta continuamente entre regiones cerebrales vinculadas estructural y funcionalmente: la red cerebral.  La conectividad funcional se define como la dependencia temporal de los patrones de actividad neuronal de regiones cerebrales separadas anatómicamente. Durante estos experimentos en estado de reposo, se instruyó a los voluntarios para que se relajaran y no pensaran en algo en particular, mientras se midió su nivel de actividad cerebral espontánea durante todo el período del experimento. Biswal y sus colegas fueron los primeros en demostrar que, durante el reposo, las regiones hemisféricas izquierda y derecha de la red motora primaria no están en silencio, sino que muestran una alta correlación entre sus series temporales de fMRI BOLD (Biswal et al., 1995, Biswal et al . , 1997 ), lo que sugiere un procesamiento de información continuo y una conectividad funcional continua entre estas regiones durante el descanso ( Biswal et al., 1997 , Cordes et al., 2000 , Greicius et al., 2003 , Lowe et al., 2000). En su estudio (ilustrado esquemáticamente en la Fig. 1 ), la serie de tiempo en estado de reposo de un vóxel en la red motora se correlacionó con la serie de tiempo en estado de reposo de todos los demás vóxeles cerebrales, lo que revela una alta correlación entre el estado neuronal espontáneo patrones de activación de estas regiones. Varios estudios han replicado estos resultados pioneros, mostrando un alto nivel de conectividad funcional entre la corteza motora hemisférica izquierda y derecha, pero también entre regiones de otras redes funcionales conocidas, como la red visual primaria, la red auditiva y las redes cognitivas de orden superior

Estos estudios marcan que durante el descanso la red cerebral no está inactiva, sino que muestra una gran cantidad de actividad espontánea que está altamente correlacionada entre múltiples regiones cerebrales . Sin embargo, el apoyo para una posible base neuronal de las señales de resonancia magnética funcional en estado de reposo proviene de la observación de que la mayoría de los patrones en estado de reposo tienden a ocurrir entre regiones del cerebro que se superponen tanto en función como en neuroanatomía, por ejemplo, regiones del sistema motor, visual y auditivo. Esta observación sugiere que las regiones del cerebro que a menudo tienen que trabajar juntas forman una red funcional durante el descanso, con un alto nivel de actividad neuronal espontánea en curso que está fuertemente correlacionada entre las regiones anatómicamente separadas que forman la red.    Estudios que informan que las señales BOLD espontáneas observadas están dominadas principalmente por frecuencias más bajas (< 0,1  Hz) con solo una contribución mínima de oscilaciones cardíacas y respiratorias más frecuentes (> 0,3  Hz) )
Esto hace que las oscilaciones de fMRI espontáneas en estado de reposo sean una medida sólida para examinar las conexiones funcionales entre las regiones del cerebro en una escala de todo el cerebro.


https://www.sciencedirect.com/science/article/abs/pii/S1053811912000614?via%3Dihub


Se ha apreciado durante al menos dos milenios que los cerebros de los humanos exhiben una actividad continua independientemente de la presencia o ausencia de cualquier comportamiento observable. Como señaló Séneca en ~  60  d.C., “ El hecho de que el cuerpo esté acostado no es razón para suponer que la mente está en paz . El descanso está... lejos de ser reparador(Séneca, 1969).  

Dada la caracterización aparentemente contradictoria de “descanso” (ver arriba) es prudente comenzar con una definición. En el contexto de la experimentación, "descanso" es una definición operativa que se refiere a una condición constante sin estímulos impuestos u otros eventos sobresalientes en el comportamiento. Los ojos pueden estar cerrados o abiertos, con o sin fijación visual. La definición operativa de "descanso" se puede generalizar para abarcar la participación en una tarea controlada siempre que toda la estructura temporal impuesta se escalone aleatoriamente con respecto a la adquisición de fMRI, por ejemplo, (Fransson, 2006). El objetivo de los experimentos en estado de reposo es capturar las propiedades estadísticas de la actividad neuronal generada endógenamente (sinónimos: espontánea, intrínseca). Por el contrario, el objetivo de los estudios relacionados con eventos es medir las respuestas evocadas o inducidas.


https://www.nature.com/articles/s41598-017-03073-5.pdf?pdf=reference


En el cerebro humano, la actividad espontánea durante el estado de reposo consiste en transiciones rápidas entre estados de redes funcionales a lo largo del tiempo, pero no se conocen los mecanismos subyacentes. Este marco teórico ha tenido un gran éxito al explicar la dinámica altamente estructurada que surge de la actividad cerebral espontánea en las llamadas redes de estado de reposo (RSN)19–21, incluso si el cerebro en reposo nunca descansa realmente20. Se ha demostrado que la actividad cerebral eficiente relacionada con la tarea se basa en la metaestabilidad de la actividad cerebral espontánea, lo que permite una exploración óptima del repertorio dinámico22, pero no se sabe si esta metaestabilidad es máximamente metaestable6. En los sistemas dinámicos, la metaestabilidad se refiere a un estado que queda fuera del estado de equilibrio natural del sistema pero que persiste durante un período de tiempo prolongado. Un ejemplo de un sistema dinámico metaestable es una competencia sin ganador23, sin embargo, la metaestabilidad puede surgir de una serie de mecanismos subyacentes y es en este sentido más amplio que usamos el término metaestabilidad. A

https://www.jneurosci.org/content/33/27/11239

Las fluctuaciones del cerebro en reposo no son aleatorias, sino que están estructuradas en patrones espaciales de actividad correlacionada en diferentes áreas del cerebro. La cuestión de cómo emerge la conectividad funcional (FC) en estado de reposo a partir de las conexiones anatómicas del cerebro ha motivado varios estudios experimentales y computacionales para comprender las relaciones estructura-función.

Las fluctuaciones cerebrales continuas en reposo (es decir, en ausencia de estimulación externa y demandas de respuesta) no son aleatorias sino que están muy estructuradas. La evidencia de numerosos experimentos electrofisiológicos y de neuroimagen demuestra que el cerebro en reposo muestra patrones espaciales de actividad correlacionada en diferentes áreas del cerebro conocidas como redes de estado de reposo (RSN)  Se ha demostrado que la estructura de correlación de las fluctuaciones BOLD espontáneas (es decir, la conectividad funcional en reposo [FC]) se relaciona con el circuito anatómico subyacente obtenido mediante imágenes de espectro de tensor de difusión (DTI/DSI) (Honey et al., 2007 ; Hagmann et al., 2008 ). Se obtuvieron resultados similares en el mono al combinar fMRI y mapas de trazadores retrógrados ( Vincent et al., 2007). Esto sugiere que las RSN surgen de correlaciones de ruido neuronal entre áreas del cerebro que están acopladas por la conectividad anatómica subyacente. Sin embargo, cómo la FC se relaciona con la conectividad anatómica y la dinámica cerebral sigue siendo una pregunta abierta y ha motivado varios estudios experimentales y de modelado de redes para comprender las relaciones estructura-función (

Una conclusión importante de estudios computacionales previos es que la relación entre la estructura anatómica y las correlaciones funcionales depende en gran medida de la dinámica local y del estado dinámico global de la red. Por lo tanto, es fundamental describir la dinámica local de las diferentes áreas del cerebro de la forma más realista posible.

https://journals.plos.org/ploscompbiol/article?id=10.1371/journal.pcbi.1000196




Tradicionalmente, la función cerebral se estudia midiendo las respuestas fisiológicas en paradigmas sensoriales, motores y cognitivos controlados. Sin embargo, incluso en reposo, en ausencia de un comportamiento abierto dirigido a un objetivo, las colecciones de regiones corticales muestran constantemente una actividad coherente en el tiempo. En los seres humanos, se ha demostrado que estas redes de estado de reposo se superponen en gran medida con las arquitecturas funcionales presentes durante la actividad conscientemente dirigida, lo que motiva la interpretación de la actividad de reposo como soñar despierto, asociación libre, flujo de conciencia y ensayo interno. En monos, se ha demostrado que fluctuaciones coherentes similares están presentes durante la anestesia profunda cuando no hay conciencia.  Ha habido un gran interés generado por la observación de redes en estado de reposo o "modo predeterminado" en el cerebro humano. Estas redes parecen estar más involucradas cuando las personas no están involucradas en un comportamiento abierto dirigido a un objetivo. También se cree que estas redes subyacen a ciertos aspectos de la introspección consciente y que son específicas de los humanos. Nuestro artículo proporciona una nueva explicación para las fluctuaciones del estado de reposo al sugerir que reflejan un principio biológico más profundo de organización y son una consecuencia de la estructura espacio-temporal de la conectividad anatómica de los primates.



Cuando los sujetos no participan activamente en una actividad mental dirigida a un objetivo, se ha sugerido que la actividad cerebral espontánea no representa simplemente "ruido", sino que implica procesos espontáneos y transitorios involucrados en imágenes y pensamientos no relacionados con la tarea [1] - [ 9 ] . Las redes de estado de reposo que no están asociadas con regiones sensoriales o motoras se han considerado como una red de "modo predeterminado" específica para el ser humano e incluyen las cortezas cingulada anterior y posterior, parietal y prefrontal medial [4] , [ 5 ] . Resultados recientes de Raichle y colaboradores [10]mostró redes similares en monos durante la anestesia profunda, lo que sugiere que esta red de modo predeterminado no es, en primer lugar, específica para el ser humano y, en segundo lugar, que trasciende los niveles de conciencia. Además, la suposición de un vínculo entre la actividad del estado de reposo y los procesos mentales se basa en gran medida "ex negativo" en los estudios de tomografía por emisión de positrones (PET) y fMRI que muestran la desactivación de la red de "modo predeterminado" en correlación con el aumento en la tarea- actividad relacionada en áreas impulsadas por los sentidos durante el comportamiento dirigido a un objetivo. 

https://www.nature.com/articles/s41598-017-03073-5

transiciones rápidas entre estados de redes funcionales a lo largo del tiempo, pero no se conocen los mecanismos subyacentes. Utilizamos modelos de redes cerebrales computacionales basadas en conectomas para revelar los principios fundamentales de cómo el cerebro humano genera actividad a gran escala observable mediante neuroimágenes no invasivas. Utilizamos datos de neuroimagen estructural y funcional para construir modelos de todo el cerebro. Con este enfoque novedoso, revelamos que el cerebro humano durante el estado de reposo opera con la máxima metaestabilidad, es decir, en un estado de máxima conmutación de red. 

https://www.sciencedirect.com/science/article/pii/S0301008213001457?via%3Dihub


particular, mientras la persona está descansando y el cuerpo está estático, el cerebro parece estar activamente involucrado, exhibiendo fluctuaciones lentas de actividad neuronal organizadas espaciotemporalmente . Los patrones de actividad cerebral observados durante el descanso tranquilo y despierto son distinguibles de los observados durante el descanso dirigido a un objetivo.comportamiento o cuando el cerebro se duermeVarios estudios han especulado sobre el vínculo entre esta actividad cerebral en reposo y los procesos cognitivos de orden superior subyacentes, como el razonamiento moral, la autoconciencia, el recuerdo de experiencias pasadas o la planificación para el futuro (Buckner et al., 2008, Lou et al., 1999 ) . 



 Sin embargo, los hallazgos de patrones cerebrales en reposo en monos anestesiados ( Vincent et al., 2007 ) y, más recientemente, en ratas ( Lu et al., 2012 ), apuntan a un origen más fundamental de las activaciones cerebrales en reposo ( Fig. 1 ) ( incluso si los animales también pueden tener una necesidad de representaciones de sí mismos). 
 
 
  A pesar de la evidencia del lado empírico, la importancia de la conectividad funcional en la actividad cerebral durante el descanso sigue siendo objeto de debate. En los últimos años, un número creciente de estudios teóricos y experimentales han tenido como objetivo investigar el origen de los patrones de correlación que definen las RSN utilizando diferentes técnicas de neuroimagen. Sin embargo, todavía no está claro si las RSN son un epifenómeno o no.Hagmann et al., 2008 , Sporns et al., 2000 ) (ver Sección 3). Es importante destacar que se ha encontrado una coincidencia notable entre la red neuroanatómica y la conectividad funcional en estado de reposo, lo que indica que las conexiones funcionales entre las áreas del cerebro pueden estar mediadas por fibras de materia blanca. Los modelos computacionales ascendentes se pueden utilizar para simular las interacciones entre áreas del cerebro en la red estructural y comparar los resultados con datos funcionales empíricos
  Entre todos los regímenes cerebrales, el estado de reposo es particularmente interesante desde la perspectiva de los sistemas dinámicos porque parece exhibir una dinámica exploratoria, en la que uno o más estados (es decir, RSN) pueden ser visitados (es decir, activados) con el tiempo, pero posteriormente desactivados. nunca fijándose en un punto fijo, resultando en un régimen no estacionario. La forma en que este tipo de comportamiento surge de la arquitectura del cerebro, en particular de la interacción entre neuronas, puede investigarse mediante modelos computacionales, reuniendo conceptos de neurofisiología y física teórica. Más detalladamente, la dinámica colectiva de un grupo de neuronas puede representarse como una unidad dinámica (gobernada por un conjunto de ecuaciones dinámicas) con un comportamiento particular en el estado espontáneo. Varias de estas unidades dinámicas pueden acoplarse de acuerdo con una conectividad cerebral realista. El repertorio dinámico del sistema (es decir, los diferentes regímenes que puede mostrar dependiendo de los parámetros del modelo) puede investigarse a través de simulaciones (rara vez es posible una solución analítica debido a la complejidad del sistema) para determinar las condiciones bajo las cuales se puede realizar una exploración. podría surgir un comportamiento similar al observado durante el descanso (ver Sección4).
  
  
  