\newglossaryentry{fitness}
{
	name=fitness,
	description={El éxito de un individuo (o alelo o genotipo en una población) en sobrevivir y reproducirse, medido por la contribución genética de ese individuo (o alelo o genotipo) a la próxima generación y las generaciones subsiguientes}
}

\newglossaryentry{epistasis}
{
	name=epistasis,
	description={El enmascaramiento del efecto fenotípico de los alelos en un gen por los alelos de otro gen. Se dice que un gen es epistático cuando su presencia suprime el efecto de un gen en otro locus. Los genes epistáticos a veces se denominan genes inhibidores debido a su efecto sobre otros genes que se describen como hipostáticos}
}

\newglossaryentry{ansatz}
{
	name=ansatz,
	description={Un término utilizado en matemáticas y física teórica para referirse a una estrategia heurística que consiste en proponer una suposición o conjetura inicial como punto de partida para resolver un problema o construir una solución. El Ansatz se basa en la intuición o en observaciones previas y busca simplificar el problema al reducir la búsqueda de soluciones posibles. En el contexto matemático, el Ansatz ayuda a abordar problemas complejos al establecer una forma o estructura específica para la solución. En física teórica, el Ansatz desempeña un papel fundamental al permitir la construcción de modelos y la resolución de ecuaciones mediante la asunción de una forma funcional particular o la imposición de restricciones sobre los parámetros involucrados}
}


\newglossaryentry{AMP}
{
	name=AMP Cíclico,
	description={Nucleótido de adenina que contiene un grupo fosfato que está esterificado en las posiciones 3'- y 5'- de la molécula de azúcar. Es un segundo mensajero y un importante regulador intracelular, que funciona como mediador de la actividad para un número de hormonas, entre las que se incluyen epinefrina, glucagón, y ACTH}
}


\newglossaryentry{automata}
{
	name={autómata},
	description={Un autómata es un modelo computacional que consiste en un conjunto de estados bien definidos, un estado inicial, un alfabeto de entrada y una función de transición. Este concepto es equivalente a otros, como autómata finito o máquina de estados finitos. En un autómata, un estado es la representación de su condición en un instante dado. El autómata comienza en el estado inicial con un conjunto de símbolos; su paso de un estado a otro se efectúa a través de la función de transición, la cual, partiendo del estado actual y un conjunto de símbolos de entrada, lo lleva al nuevo estado correspondiente} ,
	plural={autómatas},
	} 
	
	
	\newglossaryentry{regularizacion}
	{
		name={regularización},
		description={Se refiere a un proceso de introducir información adicional para solucionar un problema mal definido o para impedir el sobreajuste} ,
			} 
