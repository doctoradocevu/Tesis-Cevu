\chapter{Introducción}

Comprender la relación entre la arquitectura y la función del cerebro es una cuestión central en la neurociencia. En esa dirección, se han dedicado importantes esfuerzos en los últimos años para mapear la estructura a gran escala de distintos organismos, incluidos los intentos de construir matrices de conectividad estructural del sistema nervioso a partir de datos de imágenes.  Sin embargo,   \textquote{al igual que los genes, las conexiones estructurales por sí solas son impotentes}; por lo tanto, \textquote{el conectoma debe expresarse en actividad neuronal dinámica para ser efectivo en el comportamiento y la cognición} \cite{sporns_discovering_2012}.


Comprender el cerebro es uno de los problemas más desafiantes por los que un físico puede sentirse atraído. Como sistema con una cantidad astronómica de elementos, cada uno de los cuales se sabe que tiene muchas no linealidades, el cerebro exhibe dinámicas colectivas que en muchos aspectos se asemejan a algunos de los problemas clásicos bien estudiados en física estadística. La contradicción, y el punto provocador de estas notas, es que solo una minoría de las publicaciones en el campo hoy se preocupan por la comprensión de la dinámica del cerebro como un proceso colectivo. Los enfoques formales para estudiar fenómenos colectivos son uno de los temas clásicos en el centro de la física estadística, con aplicaciones recientes nuevas y exitosas en diversas áreas como la genética, la ecología, la informática, los entornos sociales y económicos. Si bien en todos estos campos existe una clara transferencia de métodos e ideas desde la física estadística, un flujo similar recién ha comenzado a impactar en la neurociencia.

¿cómo se coordinan entre sí esas miríadas de elementos e interacciones en criaturas vivas complejas?" o “¿cómo emerge un comportamiento coherente de tal sopa de componentes altamente heterogéneos? Una estrategia complementaria consiste en mirar problemas biológicos complejos desde una perspectiva global, cambiando el enfoque de detalles específicos de la maquinaria molecular a aspectos integrales.  Los enfoques sistémicos de la biología se basan en la evidencia de que algunos de los fenómenos más fascinantes de los sistemas vivos, como la memoria y la capacidad para resolver problemas, son fenómenos colectivos, derivados de las interacciones de muchas unidades básicas y podrían no reducirse a la comprensión de componentes elementales de forma individual (Bialek, 2018 ).  durante mucho tiempo han sido seducidos por la idea de adaptar conceptos y métodos de la mecánica estadística para arrojar luz sobre la organización a gran escala de los sistemas biológicos

Recordemos qué son los fenómenos emergentes. La emergencia se refiere a los patrones espaciotemporales colectivos inesperados exhibidos por grandes sistemas complejos. En este contexto, 'inesperado' muestra nuestra incapacidad (matemática y de otro tipo) para derivar tales patrones emergentes de las ecuaciones que describen la dinámica de las partes individuales del sistema. Como se discutió extensamente en otra parte 1 , 15Los sistemas complejos suelen ser grandes conglomerados de elementos que interactúan, cada uno de los cuales exhibe algún tipo de dinámica no lineal.


l punto importante es que incluir la complejidad en el modelo solo dará como resultado una simulación del sistema real, sin que ello implique ninguna comprensión de la complejidad. Los esfuerzos más significativos han sido los dirigidos a descubrir las condiciones en las que algo complejo emerge de la interacción de los elementos no complejos que lo constituyen


En muchos sistemas físicos, como los sistemas magnéticos o gravitacionales, ciertas características macroscópicas surgen de las interacciones de los elementos constituyentes de una manera que es impredecible incluso a partir de una comprensión perfecta del comportamiento de cada componente; esto se conoce como emergencia ( Chialvo, 2010). En el contexto del cerebro, los fenómenos emergentes abarcan el comportamiento y la cognición, que surgen de la interacción de la gran cantidad de neuronas en el cerebro. Abordar el estudio de los sistemas neuronales desde esta perspectiva implica estudiar el comportamiento neuronal a nivel de red o población: observar y comprender los comportamientos emergentes en el sistema en lugar de concentrarse en el comportamiento y las conexiones de cada neurona individual por sí sola. Si bien exhiben cierto poder computacional por sí mismas, las neuronas son realmente notables en su capacidad computacional cuando se toman en conjunto.
https://www.frontiersin.org/articles/10.3389/fncom.2021.611183/full




