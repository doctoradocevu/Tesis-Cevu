\chapter{Introducción}




muchos fenómenos altamente complejos son el re-resultado de la dinámica cooperativa y colectiva de un gran número de- Típicamente piezas muy simples-individuales. En estos patrones complejos emergentes, pueden aparecer características macro-macroscópicas reconocibles que no tienen una relación simple, visible o inmediata con las reglas microscópicas originales. El análisis matemático de estos patrones y su dinámica subyacente implica álgebra, teoría de números, teoría de grafos, lógica, topología, así como técnicas que se originan en la teoría de sistemas dinámicos.



Los patrones espacio-temporales de la actividad neural evolucionan rápidamente tanto en reposo como en respuesta a los estímulos del entorno natural.
Más recientemente, se ha prestado cada vez más atención a la relevancia funcional de las conexiones entre las regiones del cerebro. Esta tesis investiga la dinámica de los sistemas neuronales con énfasis en la conectividad funcional dirigida.
Durante mucho tiempo, la actividad cerebral espontánea se ha considerado "ruido" en las investigaciones sobre la función cerebral y, a menudo, se resta o se promedia durante el análisis de neuroimagen.
% Modelling brain dynamics by Boolean networks
Comprender la relación entre la arquitectura cerebral y la función cerebral es un tema central en la neurociencia.
El resultado más importante de este trabajo es el estudio de circuitos neuronales emergentes, es decir, configuraciones de áreas que se sincronizan en el tiempo, tanto a nivel local como global, determinando la aparición de análogos computacionales de los procesos cognitivos, que pueden o no ser similares al funcionamiento del cerebro biológico. Además, los resultados ponen en evidencia la creación de cómo el cerebro crea estructuras de comunicación a distancia. Estas estructuras tienen una organización jerárquica donde cada nivel permite el surgimiento de organizaciones cerebrales que se comportan en el siguiente nivel superior. En conjunto, estos resultados permiten comprender la interacción de las raíces dinámicas y topológicas de la dinámica cerebral multifacética.

La biología de sistemas es un campo multidisciplinario cuyo objetivo es proporcionar una comprensión a nivel de sistemas de los fenómenos biológicos al descubrir su estructura, dinámica y métodos de control [19].

 Comprender cómo los circuitos neuronales generan patrones complejos de actividad es un desafío,  y es aún más difícil construir modelos de este tipo que sigan siendo sensibles a la información sensorial. En términos matemáticos, necesitamos entender cómo un sistema puede reconciliar una rica estructura de estados internos con un alto grado de sensibilidad a las variables externas. Este problema está lejos de resolverse, pero aquí repasamos los avances que se han logrado en los últimos años.


Las leyes de potencia aparecen en una gran variedad de entornos en toda la naturaleza y, a menudo, significan que hay un proceso simple en el origen de lo que parece ser un fenómeno muy complejo. Ejemplos de la variedad de entornos en los que aparecen las leyes de potencia son la ley de Gutenberg-Richter para el tamaño de los terremotos [1] , [2] , las leyes de escala alométrica que aparecen en la biología [3] y la ley de distribución de ingresos de Paretos [4] ] .


El mapeo de la relación funcional-estructural del cerebro sigue siendo uno de los desafíos más complejos de la neurociencia moderna (Park y Friston, 2013), en parte debido a la naturaleza altamente dinámica de múltiples escalas de los procesos y estructuras del cerebro según lo observado por diferentes modalidades de medición, que conduce a dificultades técnicas y matemáticas para establecer relaciones dinámicamente invariantes entre escalas. Como resultado, aún no se comprende completamente qué función precisa surge a macroescala (medida por señales BOLD) de la arquitectura neuronal estática subyacente. De hecho, esto se basa en una relación estructura-función de muchos a uno, que es difícil de resolver y, por lo tanto, se requieren metodologías novedosas. El presente estudio aborda aspectos de esta cuestión de macroescala aprovechando los desarrollos recientes de métodos computacionales novedosos basados en datos, que eliminan estados dinámicos recurrentes de series temporales y los asocia a estructuras cerebrales óptimas, resolviendo así la estructura funcional del sistema nervioso central. llamadas redes de estado de reposo (RSN) (Raichle et al., 2001; Fox et al., 2005; Diez et al., 2015; Smitha et al.,
2017).


Las redes distribuidas de poblaciones neuronales constituyen la base del comportamiento y la cognición en animales y humanos, con complejos patrones de comunicación y señalización neuronal1. Las mejoras de las nuevas tecnologías y métodos de neuroimagen han revelado una descripción estructural y funcional detallada de los patrones de conectividad del cerebro humano2,3 y de su compleja red a gran escala. Para identificar el cerebro humano como un sistema dinámico complejo, Sporns et al.4 propusieron la noción de "conectoma". Desde entonces, varias investigaciones 5 han empleado métodos de neuroimagen, así como enfoques gráficos de vanguardia, para explorar el cerebro humano tanto en sujetos sanos como enfermos, descubriendo propiedades características en diferentes regiones del b rain 3,6-9 , presentando pruebas de la naturaleza funcional de los clusters tanto en cerebros humanos10,11 como animales12 , y realizando así "mapas de red completos de circuitos y sistemas neuronales" 13 . La disponibilidad práctica de estos métodos para los investigadores en neurociencia de redes14 , partiendo de datos sobre cómo se conectan las regiones cerebrales y adoptando el marco teórico de la ciencia de redes, permite la descripción ma hemática de la arquitectura y las funciones cerebrales. En concreto, dos de estas organizaciones matemáticas, small w orbe15 (SWN) y libre de escala16,17 (SF) 18.





Comprender la dinámica y la estructura de las redes neuronales es un desafío para biólogos, matemáticos y físicos. Las neuronas forman redes complejas de conexiones, donde las dendritas y los axones se extienden, ramifican y forman enlaces sinápticos entre las neuronas. Debido a los axones largos, la estructura de una red neuronal típica tiene propiedades de mundo pequeño. En particular, las redes neuronales en los cerebros de los mamíferos tienen longitudes de camino cortas, coeficientes de agrupamiento altos, correlaciones de grado y distribuciones de grado sesgadas  arquitectura compleja


% modelado 
%Neural field models for latent state inference: Application to large-scale neuronal recordings
Los métodos de registro neuronal a gran escala ahora nos permiten observar grandes poblaciones de neuronas individuales identificadas simultáneamente, abriendo una ventana a la dinámica de la población neuronal en los organismos vivos. Sin embargo, destilar grabaciones a gran escala para construir teorías de dinámicas colectivas emergentes sigue siendo un desafío estadístico fundamental.
Las neuronas se comunican mediante impulsos eléctricos o picos. Comprender la dinámica y la fisiología de los picos colectivos en grandes redes de neuronas es un desafío central en la neurociencia moderna, con un inmenso potencial clínico y de traducción.

Conectar la dinámica de una sola neurona con el comportamiento de la población ha sido el foco central de la investigación dentro de la comunidad de neurociencia teórica durante las últimas cuatro décadas.



% aplication of percolation theory
Es un hecho de la vida, que es tan desafiante para la mente del científico como frustrante para sus aspiraciones, que la naturaleza está desordenada. Sólo en el supermercado de los teóricos podemos comprar sistemas limpios, puros, perfectamente caracterizados y geométricamente inmaculados. Un ingeniero trabaja en un mundo de compuestos y mezclas (cuánto más el biólogo). Incluso el experimentador que se enfoca en la más pura de las sustancias, ejemplificada por cristales cuidadosamente cultivados, rara vez puede escapar de los efectos de los defectos, trazas de impurezas y límites finitos. Hay pocos conceptos en la ciencia más elegantes de contemplar que una red cristalina infinita, perfectamente periódica, y pocos sistemas tan alejados de la realidad experimental. Por lo tanto, estamos obligados a aceptar estructuras desordenadas; la variación en forma y constitución a menudo está tan mal caracterizada que debemos considerarla aleatoria si queremos describirla: aparente aleatoriedad en la morfología del sistema. La morfología de un sistema tiene dos aspectos principales: la topología, la interconexión de los elementos microscópicos individuales del sistema, y la geometría, la forma y el tamaño de estos elementos individuales. Pero quizás la razón más importante del rápido desarrollo de la física estadística de los sistemas desordenados es que se ha apreciado el papel de la interconectividad de los elementos microscópicos de un sistema desordenado y su efecto sobre las propiedades macroscópicas del sistema. Esto ha sido posible a través del desarrollo y aplicación de la teoría de la percolación, el tema de este libro.


%https://www.liebertpub.com/doi/10.1089/brain.2012.0120
El cerebro contiene ~10 11 neuronas unidas por ~10 15 conexiones, y cada neurona tiene entradas del orden de 10 5 . Los patrones de interacción neuronal complejos y altamente no lineales son poco conocidos, y el número de grados de libertad de un modelo microscópico que intenta describir cada neurona, cada conexión y cada interacción es astronómicamente grande y, por lo tanto, demasiado alto para encajarlo directamente con el datos macroscópicos registrados. La brecha entre las fuentes microscópicas de los potenciales del cuero cabelludo en las membranas celulares y los potenciales macroscópicos registrados puede salvarse mediante una descripción mesoscópica intermedia (Nunez y Silberstein, 2000) .).



Comprender la relación entre la arquitectura y la función del cerebro es una cuestión central en la neurociencia. En esa dirección, se han dedicado importantes esfuerzos en los últimos años para mapear la estructura a gran escala de distintos organismos, incluidos los intentos de construir matrices de conectividad estructural del sistema nervioso a partir de datos de imágenes.  Sin embargo,   \textquote{al igual que los genes, las conexiones estructurales por sí solas son impotentes}; por lo tanto, \textquote{el conectoma debe expresarse en actividad neuronal dinámica para ser efectivo en el comportamiento y la cognición} \cite{sporns_discovering_2012}.


\url{https://sebastianrisi.com/self_assembling_ai/}

no de los aspectos más fascinantes de la naturaleza es que grupos con millones o incluso billones de elementos pueden autoensamblarse en formas complejas basadas solo en interacciones locales y mostrar lo que se llama un tipo colectivo de inteligencia. Por ejemplo, las hormigas pueden unirse para crear puentes o balsas para navegar por terrenos difíciles, las termitas pueden construir nidos de varios metros de altura sin un plan impuesto desde el exterior, y miles de abejas trabajan juntas como un todo integrado para tomar decisiones precisas sobre cuándo buscar alimento o un nuevo nido. Sorprendentemente, lograr estas increíbles habilidades es el resultado de seguir reglas de comportamiento relativamente simples y a través de un proceso de autoorganización, que Camazine et al. (2001) definen como:

“ Como un proceso en el que el patrón a nivel global de un sistema surge únicamente de numerosas interacciones entre los componentes de nivel inferior del sistema. Además, las reglas que especifican las interacciones entre los componentes del sistema se ejecutan utilizando solo información local, sin referencia al patrón global. En resumen, el patrón es una propiedad emergente del sistema en lugar de ser impuesto al sistema por una influencia de orden externa. “

Comprender el cerebro es uno de los problemas más desafiantes por los que un físico puede sentirse atraído. Como sistema con una cantidad astronómica de elementos, cada uno de los cuales se sabe que tiene muchas no linealidades, el cerebro exhibe dinámicas colectivas que en muchos aspectos se asemejan a algunos de los problemas clásicos bien estudiados en física estadística. La contradicción, y el punto provocador de estas notas, es que solo una minoría de las publicaciones en el campo hoy se preocupan por la comprensión de la dinámica del cerebro como un proceso colectivo. Los enfoques formales para estudiar fenómenos colectivos son uno de los temas clásicos en el centro de la física estadística, con aplicaciones recientes nuevas y exitosas en diversas áreas como la genética, la ecología, la informática, los entornos sociales y económicos. Si bien en todos estos campos existe una clara transferencia de métodos e ideas desde la física estadística, un flujo similar recién ha comenzado a impactar en la neurociencia.

¿cómo se coordinan entre sí esas miríadas de elementos e interacciones en criaturas vivas complejas?" o “¿cómo emerge un comportamiento coherente de tal sopa de componentes altamente heterogéneos? Una estrategia complementaria consiste en mirar problemas biológicos complejos desde una perspectiva global, cambiando el enfoque de detalles específicos de la maquinaria molecular a aspectos integrales.  Los enfoques sistémicos de la biología se basan en la evidencia de que algunos de los fenómenos más fascinantes de los sistemas vivos, como la memoria y la capacidad para resolver problemas, son fenómenos colectivos, derivados de las interacciones de muchas unidades básicas y podrían no reducirse a la comprensión de componentes elementales de forma individual (Bialek, 2018 ).  durante mucho tiempo han sido seducidos por la idea de adaptar conceptos y métodos de la mecánica estadística para arrojar luz sobre la organización a gran escala de los sistemas biológicos

Recordemos qué son los fenómenos emergentes. La emergencia se refiere a los patrones espaciotemporales colectivos inesperados exhibidos por grandes sistemas complejos. En este contexto, 'inesperado' muestra nuestra incapacidad (matemática y de otro tipo) para derivar tales patrones emergentes de las ecuaciones que describen la dinámica de las partes individuales del sistema. Como se discutió extensamente en otra parte 1 , 15Los sistemas complejos suelen ser grandes conglomerados de elementos que interactúan, cada uno de los cuales exhibe algún tipo de dinámica no lineal.


l punto importante es que incluir la complejidad en el modelo solo dará como resultado una simulación del sistema real, sin que ello implique ninguna comprensión de la complejidad. Los esfuerzos más significativos han sido los dirigidos a descubrir las condiciones en las que algo complejo emerge de la interacción de los elementos no complejos que lo constituyen


En muchos sistemas físicos, como los sistemas magnéticos o gravitacionales, ciertas características macroscópicas surgen de las interacciones de los elementos constituyentes de una manera que es impredecible incluso a partir de una comprensión perfecta del comportamiento de cada componente; esto se conoce como emergencia ( Chialvo, 2010). En el contexto del cerebro, los fenómenos emergentes abarcan el comportamiento y la cognición, que surgen de la interacción de la gran cantidad de neuronas en el cerebro. Abordar el estudio de los sistemas neuronales desde esta perspectiva implica estudiar el comportamiento neuronal a nivel de red o población: observar y comprender los comportamientos emergentes en el sistema en lugar de concentrarse en el comportamiento y las conexiones de cada neurona individual por sí sola. Si bien exhiben cierto poder computacional por sí mismas, las neuronas son realmente notables en su capacidad computacional cuando se toman en conjunto.
https://www.frontiersin.org/articles/10.3389/fncom.2021.611183/full




